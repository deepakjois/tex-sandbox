\documentclass[a4paper]{article}

\usepackage[margin=1in,centering]{geometry}
\usepackage{fontspec}
\usepackage{array}
\usepackage{psvectorian}
\usepackage{enumitem}
\usepackage{csvsimple}
\usepackage{multicol}
\usepackage{bidi}

\newfontfamily\notourdu[Script=Arabic,Language=Urdu,WordSpace=5]{Noto Nastaliq Urdu}
\newfontfamily\itfhindi[Script=Devanagari]{ITF Devanagari}
\newfontfamily\baskerville{Baskerville}
\newfontfamily\titlefont[Script=Arabic,Language=Urdu,FakeBold=1.5]{Noto Nastaliq Urdu}

\newcommand{\deco}{\psvectorian[height=0.75cm]{71}}

\newcommand{\subtitle}[1]{\begingroup\fontsize{18pt}{12pt}\titlefont #1\endgroup\nopagebreak}

\newcommand{\sethindiurduspelling}[2]{\itfhindi #1 \> \notourdu\RLE{#2}\\}

\begin{document}
% ---------------- COVER PAGE BEGINS  ----------------
\thispagestyle{empty}
\begin{center}
\itfhindi
\Huge \textbf{महात्मा गांधी अंतरराष्ट्रीय हिंदी विश्वविद्यालय, वर्धा}
\vskip 20pt
\huge भारतीय एवं विदेशी भाषा प्रगत अध्ययन केन्द्र
\vskip 10pt
भाषा विद्यापीठ
\vskip 42pt
\deco\deco\deco
\vskip 42pt
\huge \textbf{टर्म पेपर}
\vskip 20pt
\vskip 10pt
उर्दू मश्क़ किताबचह
\vskip 10pt
\RL{\notourdu اردو مشق کتابچہ}
\vskip 42pt
\deco\deco\deco
\end{center}
\vskip 42pt
\itfhindi
\setlength{\tabcolsep}{0pt}
\noindent\begin{tabular}{>{\noindent\centering}p{225pt}>{\noindent\centering}p{225pt}}
\huge \textbf{प्रस्तुतकर्ता}
\vskip 10pt
\LARGE दीपक जोईस\\
उर्दू भाषा में एडवांस्ट डिप्लोमा\\प्रथम छमाही\\2015-16
&
\huge \textbf{मार्गदर्शक}\\
\vskip 10pt
\LARGE सेराज अहमद अंसारी\\
सहायक प्रोफेसर\\
भारतीय एवं विदेशी भाषा प्रगत अध्ययन केन्द्र\\
\end{tabular}
\clearpage
% ---------------- COVER PAGE ENDS ----------------
\newpage\null\thispagestyle{empty}\newpage
% ---------------- PAGE INTENTIONALLY LEFT BLANK ----------------
\pagenumbering{arabic}
\setlength{\parindent}{0pt}
\setlength{\parskip}{24pt plus 10pt minus 10pt}
\setRTL
\begin{center}
\fontsize{24pt}{18pt}{\titlefont پیشِ لفظ}
\end{center}

\fontsize{12pt}{12pt}\linespread{2.5}\notourdu
  اس کتابچہ میں اردو سیکھنے کے لئے کچھ وسائل مشتمل ہیں۔ یہ کتابچہ خاص طور پر اُن کو نظر میں رکھ کر بنای گئی ہے جو  اردو رسم الخط کو پڑھنا سیکھ گئے ہیں اور اب مشق کرنا چاہتے ہیں۔ اِس کتابچہ کے ابواب اور اُنکی مختصر تفصیل مندرجہ ذیل ہے۔

۱) ہندی اور اردو میں کچھ الفاظ:  تلفّظ کے حساب سے اردو کے وہ الفاظ جن کو لکھنے میں نوآموز طلبہ و طالبات کو پریشانی ہوتی ہے۔

۲) اردو واحد جمع: واحد جمع کو ہندی میں
{\itfhindi एकवचन-बहुवचन}
اور انگریزی میں
{\baskerville Singular-Plural}
بھی کہتے ہیں۔

۳) کچھ اردو الفاظ اور اُن کے معنی: اردو کے قریباً ۲۰۰ الفاظ اور اُن کے ہندی میں معنی۔

امید کی جاتی ہے کہ یہ کتابچہ اردو  سیکھنے والوں کے 
لئے فائدہ مند ثابت ہوگی۔

– دیپک جویس
\clearpage
\begin{center}
\fontsize{24pt}{18pt}{\titlefont ۱ – ہندی اور اردو میں کچھ الفاظ}
\end{center}

\setLTR
\fontsize{12pt}{12pt}\linespread{2.5}\notourdu
\setlength\columnsep{20pt}
\setlength\multicolbaselineskip{0pt}
\begin{multicols}{3}
\begin{tabbing}
\hspace*{2cm}\=\hspace*{2.5cm}\= \kill
\sethindiurduspelling{नक़्शा}{نقشہ}
\sethindiurduspelling{काग़ज़}{کاغذ}
\sethindiurduspelling{नारा}{نعرہ}
\sethindiurduspelling{इत्तफ़ाक़}{اتفاق}
\sethindiurduspelling{ज़ाहिर}{ظاہر}
\sethindiurduspelling{यानी}{یعنی}
\sethindiurduspelling{साबित}{ثابت}
\sethindiurduspelling{सही}{صحیح}
\sethindiurduspelling{क़िश्त}{قسط}
\sethindiurduspelling{मुलाक़ात}{ملاقات}
\sethindiurduspelling{तय}{طے}
\sethindiurduspelling{वास्ते}{واسطے}
\sethindiurduspelling{राज़ी}{راضی}
\sethindiurduspelling{तबला}{طبلہ}
\sethindiurduspelling{ग़लत}{غلط}
\sethindiurduspelling{वादा}{وعدہ}
\sethindiurduspelling{ज़रूर}{ضرور}
\sethindiurduspelling{नुस्ख़ा}{نسخہ}
\sethindiurduspelling{ख़ास}{خاص}
\sethindiurduspelling{शौक़}{شوق}
\sethindiurduspelling{इस्तमाल}{استعمال}
\sethindiurduspelling{हिस्सा}{حصّہ}
\sethindiurduspelling{बग़ैर}{بغیر}
\sethindiurduspelling{आदत}{عادت}
\sethindiurduspelling{तहज़ीब}{تہذیب}
\sethindiurduspelling{ज़रिया}{ذریعہ}
\sethindiurduspelling{ईसवी}{عیسوی}
\sethindiurduspelling{लिहाज़ा}{لہٰذا}
\sethindiurduspelling{इज़हार}{اظہار}
\sethindiurduspelling{मक़सद}{مقصد}
\sethindiurduspelling{एतबार}{اعتبار}
\sethindiurduspelling{दावा}{دعویٰ}
\sethindiurduspelling{इलाक़ा}{علاقہ}
\sethindiurduspelling{मौक़ा}{موقہ}
\sethindiurduspelling{अलावा}{علاوہ}
\sethindiurduspelling{उसूल}{اصول}
\sethindiurduspelling{मानी}{معنی}
\sethindiurduspelling{लुत्फ}{لطف}
\sethindiurduspelling{हैसियत}{حیثیت}
\sethindiurduspelling{वाक़्यात}{واقعات}
\sethindiurduspelling{ताक़त}{طاقت}
\sethindiurduspelling{क़िस्से}{قصے}
\sethindiurduspelling{तौर}{طور}
\sethindiurduspelling{दरअसल}{دراصل}
\sethindiurduspelling{तस्वीर}{تصویر}
\sethindiurduspelling{दायरा}{دائرہ}
\sethindiurduspelling{माहौल}{ماحول}
\sethindiurduspelling{उमदा}{عمدہ}
\sethindiurduspelling{शोले}{شعلے}
\sethindiurduspelling{तबक़ा}{تبقہ}
\sethindiurduspelling{इन्साफ़}{انصاف}
\sethindiurduspelling{पुख़ता}{پختہ}
\sethindiurduspelling{मुताबिक़}{مطابق}
\sethindiurduspelling{नसीब}{نصیب}
\sethindiurduspelling{सलीक़ा}{سلیقہ}
\sethindiurduspelling{मालूम}{معلوم}
\sethindiurduspelling{हाल}{حال}
\sethindiurduspelling{मसलन}{مثلاً}
\sethindiurduspelling{साफ़}{صاف}
\sethindiurduspelling{नसीहत}{نصیحت}
\sethindiurduspelling{मज़ाक़}{مذاق}
\sethindiurduspelling{बहस}{بحث}
\end{tabbing}
\end{multicols}
\clearpage
\setRTL
\begin{center}
{\fontsize{24pt}{18pt}\titlefont ۲ – اردو واحد جمع}
\end{center}

\RTLmulticolcolumns
\fontsize{12pt}{12pt}\linespread{2.5}\notourdu
\setlength\columnsep{20pt}
\begin{multicols}{3}
\begin{tabbing}
\hspace*{2cm}\=\hspace*{2.5cm}\= \kill
اسم \> اسماء\\
ادب \> آداب\\
امر \> امور\\
اصل \> اصول\\
آخر \> اواخر\\
اثر \> آثار\\
افق \> آفاق\\
امت \> امم\\
اب \> آباء\\
آفت \> آفات\\
ادیب \> ادبار\\
امیر \> امراء\\
استاد \> اساتذہ\\
الم \> آلام\\
اہل \> اہالی\\
اسلوب \> اسالیب\\
آلہ \> آلات\\
اقارب \> اقرب\\
احسان \> احسانات\\
ارض \> آراضی\\
امام \> ائمہ\\
آیت \> آیات\\
اکبر \> اکابر\\
اشتہار \> اشتہارات\\
اصلاح \> اصلاحات\\
باب \> ابواب\\
بحر \> بحور\\
بیگم \> بیگمات\\
تاریخ \> تاریخ\\
باغ \> باغات\\
تاجر \> تجار\\
تحفہ \> تہائف\\
ترجمہ \> تراجم\\
تصنیف \> تصانیف\\
تصویر \> تصاویر\\
تکلییف \> تکالیف\\
تفصیل \> تفاصیل\\
تفسیر \> تفاسیر\\
تقدیر \> تقادیر\\
تقصیر \> تقاصیر\\
تفکر \> تفکرات\\
تدبیر \> تدابیر\\
ثمر \> اثمار\\
جرم \> جرائم\\
جسم \> اجسام\\
جِرم \> اجرام\\
جد \> اجداد\\
جذبہ \> جذبات\\
جزترہ \> جزائر\\
جاہل \> جہلا\\
جانب \> جوانب\\
جن \> جنات\\
جوہر \> جواہر\\
حادثہ \> حادثات، حوادث\\
حاجی \> حجاج\\
حکم \> احکام\\
حد \> حدود\\
حرف \> حروف\\
حکیم \> حکمار\\
حبیب \> احباب\\
حال \> احوال\\
حق \> حقوق\\
حُسن \> محاسن\\
حاکم \> حکام\\
حاشیہ \> حواشی\\
حقیقت \> حقائق\\
حضرت \> حضرات\\
حافظ \> حفاظ\\
حیوان \> حیوانات\\
حاجت \> حاجات\\
حرکت \> حرکات\\
حکایت \> حکایات\\
حصّہ \> حصص\\
خط \> خطوط\\
خبر \> اخبار\\
خادم \> خدام\\
خدمت \> خدمات\\
خزانہ \> خزائن\\
خلق \> اخلاق\\
خَلق \> خَلائق\\
خیال \> خیالات\\
خلیفہ \> خلفاء\\
خاتون \> خواتین\\
خصوصیت \> خصوصیات\\
خصلت \> خصائل\\
دوا \> ادویہ\\
دفینہ \> دفائن\\
دہر \> دہور\\
دور \> ادوار\\
درجہ \> درجات\\
دیوان \> دواوین\\
دستور \> دساتیر\\
دفتر \> دفاتر\\
دین \> ادیان\\
دلیل \> دلائل\\
دیہہ \> دیہات\\
زرہ \> زرات\\
زخیرہ \> زخائر\\
زریعہ \> زرائع\\
زکر \> ازکار\\
زہن \> ازہان\\
رسول \> رُسل\\
رعیت \> رعایا\\
رفیق \> رفقاء\\
رکن \> ارکان\\
رئیس \> رؤسا\\
رب \> ارباب\\
رزق \> ارزاق\\
رسالہ \> رسائل\\
رسم \> رسوم\\
رقم \> رقوم\\
رقم (لکھنا) \> ارقام\\
روایت \> روایت\\
روضہ \> ریاض\\
روح \> ارواح\\
سبب \> اسباب\\
سماء \> سماوات\\
سلطان \> سلاطین\\
سید \> سادات\\
سفیر \> سفراء\\
نفسُ \> انفاس\\
نفس \> نفوس\\
نہر \> انہار\\
نقش \> نقوش\\
نجم \> نجوم، انجم\\
نبی \> انبیاء\\
نعمت \> نعم\\
ناصح \> نصحا\\
ناسر \> انسار\\
نتیجہ \> نتائج\\
نادر \> نوادر\\
لفیحت \> لفائح\\
وہم \> اوہام\\
وقف \> اوقاف\\
ورق \> اوراق\\
وجہہ \> وجوہ\\
وقت \> اوقات\\
وصف \> اوصاف\\
وزن \> اوزان\\
ولی \> اولیاء\\
وقیل \> وقلا\\
وارث \> ورثاء\\
وزیر \> وزراء\\
وظیفہ \> وظائف\\
واقعہ \> واقعات\\
ہمت \> ہمم\\
یتم \> یتامیٰ\\
یوم \> ایّام\\
سلف \> اسلاف\\
سبق \> اسباق\\
ساحل \> سواحل\\
شاعر \> شوراء\\
شاہد \> شواہد\\
شکایت \> شکایات\\
شریف \> شرفا\\
شخص \> اشخاص\\
شریک \> شرکاء\\
شک \> شکوک\\
شدیف \> شدائد\\
شیخ \> شیوخ\\
شعر \> اشعار\\
شجر \> اشجار\\
شیطان \> شیاطین\\
شے \> اشیاء\\
شکل \> اشکال\\
شہید \> شہدا\\
صالح \> صلحا\\
صدقہ \> صدقات\\
صنم \> اصنام\\
صندید \> صنادید\\
صنف \> اصناف\\
صورت \> صور\\
صوم \> صیام\\
صبح \> صبوح\\
صاحب \> اصحاب\\
صدمہ \> صدمات\\
صفت \> صفات\\
صحرا \> صحارا\\
صفحہ \> صفحات\\
ضد \> اضداد\\
ضلع \> اضلاع\\
ضابتہ \> ضوابط\\
طرف \> اطراف\\
طبیب \> اطبّاء\\
طبیعت \> طبائع\\
طالب \> طلبہ\\
طائر \> طیور\\
طفل \> اطفال\\
طبقہ \> طبقات\\
طلسم \> طلسمات\\
طور \> اطوار\\
ظرف \> ظروف\\
عزیز \> اعزّا\\
عندلیب \> عنادل\\
عقیدہ \> عقائد\\
عنصر \> عناصر\\
علامت \> علامات\\
غم \> غموم\\
غریب \> غرباء\\
عبد \> عباد\\
عضو \> اعضاء\\
ظلمت \> ظلمات\\
عادت \> عادات\\
عدو \> اعدا\\
عِلم \> علوم\\
علم \> اعلام\\
عالم \> عالمین\\
عِالم \> علماء\\
عارضہ \> عوارض\\
عمر \> اعمار\\
عیب \> عیوب\\
عاشق \> عُشاق\\
عزم \> عزائم\\
عام \> عوام\\
عقل \> عقول\\
عمل \> اعمال\\
عدد \> اعداد\\
عہد \> عہود\\
عجیب \> عجائب\\
عقرب \> عقارب\\
غلط \> اغلات\\
غزا \> اغزیہ\\
غلام \> غلمان\\
غیر \> اغیار\\
غرض \> اغراض\\
فرد \> افراد\\
فعل \> افعال\\
فوج \> افواج\\
فتح \> فتوح\\
فضل \> افضال\\
فلک \> افلاک\\
فہم \> افہام\\
فرش \> فروش\\
فرمان \> فرمین\\
فتویٰ \> فتاویٰ\\
فن \> فنون\\
فساد \> فسادات\\
فقیر \> فقراء\\
فقرہ \> فقرات\\
فصیح \> فصحاء\\
فاضل \> فضلا\\
فاعدہ \> فواعد\\
قدیم \> قدما\\
قدد \> اقداد\\
قسط \> اقساط\\
قول \> اقوال\\
قبر \> قبور\\
قلب \> قلوب\\
قطرہ \> قطرات\\
قوت \> قویٰ\\
قدم \> اقدام\\
قسم \> اقسام\\
قوم \> اقوام\\
قید \> قیود\\
قصّہ \> قصص\\
قاضی \> قضات\\
قبیلہ \> قبائل\\
قانون \> قوانین\\
قرینہ \> قرائن\\
قصیدہ \> قصائد\\
کافر \> کفار\\
کیفیت \> کیفیات\\
کبیر \> کبار\\
کلمہ \> کلمات\\
کوکب \> کواکب\\
کتاب \> کتب\\
کمال \> کمالات\\
لغت \> لغات\\
لیل \> لیالی\\
لحد \> لحود\\
لطف \> الطاف\\
لزت \> لزّات\\
لطیفہ \> لطائف\\
لمحہ \> لمحات\\
لفظ \> الفاظ\\
لازمہ \> لوازم\\
لقب \> القاب\\
مدرسہ \> مدارس\\
ملت \> ملل\\
مصیبت \> مصائب\\
مراسلہ \> مراسلات\\
ممکن \> ممکنات\\
مقام \> مقامات\\
معجزہ \> معجزات\\
مثل \> امثال\\
مُلک \> ممالک\\
مِلک \> اِملاک\\
مَلکَ \> مَلائک\\
مَلِک \> ملوک\\
موت \> اموات\\
مجلس \> مجالس\\
مزہب \> مزاہب\\
مرض \> امراض\\
مطلب \> مطالب\\
مہم \> مہمات\\
معاملہ \> معاملات\\
موج \> امواج\\
محنت \> محِن\\
محفل \> محافل\\
مرتبہ \> مراتب\\
مرحلہ \> مراحل\\
مسجد \> مساجد\\
مشغلہ \> مشاغل\\
مقصد \> مقاصد\\
منظر \> مناظر\\
ماکول \> ماکولات\\
مسکین \> مساکین\\
مرثیہ \> مراثی\\
مقبرہ \> مقابر\\
منزل \> منازل\\
موقع \> مواقع\\
مشہور \> مشاہیر\\
نور \> انوار\\
نکتہ \> نکات\\
\end{tabbing}
\end{multicols}
\clearpage
\begin{center}
{\fontsize{24pt}{18pt}\titlefont ۳ – کچھ اردو الفاظ اور اُنکے معنی}
\end{center}
\fontsize{12pt}{12pt}\linespread{1.2}\itfhindi
\newcommand{\seturdulabel}[1]{\item[\RLE{\notourdu #1}]\hfill\\}
\setLTR
\RTLmulticolcolumns
\begin{multicols}{3}
\setlist[description]{font=\normalfont}
\begin{description}[listparindent=0pt,leftmargin=0pt]
\csvreader[head=false,separator=tab]{meanings.txt}{1=\word,2=\meaning}%
{%
\seturdulabel{\word}
\meaning
}%
\end{description}
\end{multicols}
\end{document}
