\documentclass[12pt,a4paper,titlepage]{report}
\begin{document}
\chapter{Introduction}\label{introduction}

\section{Brief Overview of India's Caste
System}\label{brief-overview-of-indias-caste-system}

\subsection{The Origins of Caste}\label{the-origins-of-caste}

The institution of caste forms the basis India's social structure. The
caste system divides Indian society into thousands of castes and
sub-castes, based on birth, which form a ``graded hierarchy''. Members
of these endogamous groups are bound by rules which dictate how they
interact with and relate to members of other castes. Within the system,
occupations and property rights are hereditary and compulsory. Violation
of caste norms invites social and economic ostracism, which acts as a
regulatory mechanism and keeps the hierarchy in place. (Thorat)

The caste system has its origins in the fourfold Varna system, first
mentioned as a divinely ordained hierarchy in the most ancient of Hindu
texts, the Rig Veda, which divides human beings into four categories
based on their birth. At the very helm of the order are the Brahmins or
priests, followed by the Kshatriyas or warriors, then the Vaishyas who
are traders and farmers, and finally the Shudras, the labouring masses
who are slaves to the other three. The `untouchables' or Dalits do not
even find a place within this order, and since their incorporation into
the Hindu fold have been considered the lowest of all and been subjected
to the worst forms of discrimination. B.R. Ambedkar described this
system of social stratification as being ``an ascending scale of
reverence and a descending scale of contempt.''.

Over time, this division transformed into a far more complex system
consisting of thousands of graded, endogamous groups. The ideology
underlying this division, which could be called `Brahmanism', was
perpetuated, enforced and reinforced in society through violence and
coercion by the most powerful castes, with the ideological support of
Hindu religious scriptures which never fail to emphasise the importance
of maintaining caste order and following one's (caste) `dharma'.
(Riddles of Hinduism, Debrahmanising History) This religious discourse
formed the ideological basis for a society in which the lower castes
were kept in slavery-like bondage and assigned varying levels of
``pollutedness'', which was seen as justification for the brutal forms
of discrimination and oppression they were subjected to.

These caste groups eventually evolved into ``watertight compartments''
(Ambedkar) of occupation-based communities. So much so, that in India
the words commonly used by many people to refer to professionals such as
goldsmiths (called Sonar in Hindi, for example), barbers (Nai), grocers
(Bania), washermen (Dhobi), carpenters (Badhai), gardeners (Mali),
shoemakers (Mochi), etc. are actually names of individual castes whose
members have been engaged in those professions for countless
generations.

As a result of the imposition of this system, the vast majority of
people falling within the caste order have lacked any kind of social or
economic mobility for a period of close to two thousand years. Caste has
become so ingrained in the Indian psyche that caste-divisions have
continued to persist, albeit in a weakened form, even when large
sections of people have adopted new religions such as Islam and
Christianity.

\subsection{Caste in Transition}\label{caste-in-transition}

There have, of course, been many transformations in the traditional
system over the course of history. Many profound changes have occurred
in recent times, which have challenged the hegemony of the upper-castes.

Constitutionally mandated reservations in educational and administrative
public institutions, which were brought into force at after
independence, have allowed a small but significant section of Scheduled
Castes (SC, a constitutional term roughly equivalent to the category of
Dalit), Scheduled Tribes (ST, official term for India's indigenous
people, also known as Adivasis) and more recently, the Other Backward
Classes (OBC, official term roughly equivalent to the category of
Shudra) to acquire higher education and high and medium level government
jobs, which has created a new dalit-bahujan middle class.

The political mobilisation of the lower castes which began in the
colonial period has only gained strength in independent India.
Dalit-bahujan political parties have found remarkable success in many
parts of the country, and have successfully challenged upper-caste
hegemony. Political reservations, as well as the large population of
lower-castes, ensure that no political party can afford to completely
disregard their interests.

The introduction of modern economic regulations and policies in the
British period, carried forward by the government in independent India
have brought about structural changes in the Indian economy, which have
to an extent diluted the monopoly of a very small set of castes on trade
and business. The opening up of the Indian economy to the international
market in the mid 1980s has accelerated this process.

Industrialisation, urbanisation and technological progress have produced
new kinds of jobs and modified existing ones. These have created
opportunities for urban migration, which has allowed many lower-caste
people to escape the oppressive social order of their villages and move
to towns and cities, where they are able to live with relatively greater
dignity and freedom.

In spite of these changes, economic data from India suggests that the
underlying patterns of educational achievement, employment, land and
asset ownership and standards of living in Indian society continue to be
determined by caste. It is a well established fact that the
opportunities afforded to an individual for educational, economic and
political advancement in Indian society still depend to a very large
extent on the caste into which they are born. (Thorat)

\subsection{The Economic Consequences of
Caste}\label{the-economic-consequences-of-caste}

The caste system obviously has far-reaching economic consequences. In
the words of Ambedkar:

\begin{quote}
``Caste System is not merely division of labour. It is also a division
of labourers\ldots{}{[}it{]} is an hierarchy in which the divisions of
labourers are graded one above the other\ldots{}As an economic
organization caste is\ldots{} a harmful institution, inasmuch as, it
involves the subordination of man's natural powers and inclinations to
the exigencies of social rules.''
\end{quote}

Caste undermines any possibility of achieving efficiency and optimality
in economic processes as it disrupts the forces which push an economic
system in such a direction. Its imposition of occupation based on birth
restricts access to different modes of living and means of livelihood,
often forcing people into jobs they have no inclination for, or
compelling them to stay unemployed rather than take up occupations not
compatible with their caste. (Ambedkar)

Caste also divides people into antagonistic groups, foreclosing the
possibility of solidarity across the lines of class or gender. It is,
therefore, believed by many to be largely responsible for the startling
levels of poverty and deprivation in India, which are completely out of
proportion to the economic resources and capacity of the Indian state.

\section{Brief Overview of the India's Retail
Economy}\label{brief-overview-of-the-indias-retail-economy}

\subsection{India's Retail Economy}\label{indias-retail-economy}

Retail is defined as the sale of goods and services directly to
consumers. Retail trade may be contrasted with wholesale trade, which is
the sale of goods in bulk for retail or other commercial purposes.
Important retail categories include food and beverages, clothing,
healthcare, jewellery, electronics and electrical appliances, and
personal care products.

In India, the retail market is one of the biggest sectors of the economy
in terms of its economic value, and generates about 10\% of India's
gross domestic product and 8\% of the employment. There are estimated to
be 12 to 15 million retail outlets in India, and only 4\% of them are
larger than 500 square feet in size. About 92\% of all retailing in
India happens in the unorganised sector, through unregistered small
shops and vendors. (KPMG)

The most common retail outlet for the purchase of daily groceries in
India is still the small owner-manned kirana shop, though in recent
times, large convenience and grocery store chains have also started to
emerge as popular outlets among the more affluent residents of bigger
cities.

(http://indiainbusiness.nic.in/newdesign/index.php?param=industryservices\_landing/383/3)
https://www.kpmg.com/IN/en/IssuesAndInsights/ArticlesPublications/Documents/BBG-Retail.pdf

India's retail sector thus largely remains outside the bounds of the
regulatory powers of the state. Though the proliferation of industrially
produced consumer goods in the last few decades has helped the retail
industry expand and reach new segments of the Indian population, its
functioning is still governed more by traditional social and economic
norms than modern economic policy.

In recent years, the Indian government has allowed foreign direct
investment of upto 51\% in single and multi-brand retail stores in
India. This has the potential to drastically alter the current face of
Indian retail in the years to come.

\subsection{Caste and Trade in India}\label{caste-and-trade-in-india}

Trade in India, including retailing, has always been associated with a
few specific castes. This association is obviously rooted in the caste
system which designates the tasks of business, trade and money-lending
to castes lying in the Vaishya varna. On the other hand, historically,
the lower castes have been assigned the role of the labouring classes
and have had no right to own property. Thus, though the lower castes
have always formed the backbone of the economy as the producers of
commodities, due to the unsurmountable barriers created by
caste-divisions, distribution and wealth accumulation have been the
preserve of the upper-castes.

Anyone who has shopped in an Indian bazaar would know that the
traditional trading castes continue to dominate the market even in the
21st century. Baniyas, Jains, Sindhis, Khatris, and other castes which
have been traditionally associated with the occupation of trade, can be
found manning most kirana, clothing, jewellery and sweet shops in most
parts of the country. Additionally, caste-based business associations,
formed to protect the economic interests of certain castes or groups of
castes, have tremendous influence and often play a crucial role in
determining the terms of trade and employment. (Harriss-White) Many
traditional traders, true to their caste dharma, also indulge in
small-scale money-lending.

Chandra Bhan Prasad and Milind Kamble, in an article written in support
of foreign direct investment in the retail sector, state:

\begin{quote}
``Amongst segments of the Indian economy, trading is still regulated by
tradition. Retail in particular is deeply rooted in antiquity. Not
surprisingly, traders also double as moneylenders. This parallel banking
system that traditional traders practise hurts the economy, discourages
new players, and most often, blocks the circulation of currency.''
\end{quote}

(http://timesofindia.indiatimes.com/edit-page/To-empower-dalits-do-away-with-Indias-antiquated-retail-trading-system/articleshow/17482382.cms)

Not surprisingly, in India's present day capitalist class - which has
grown from strength to strength since the advent of neoliberalism - the
richest and most powerful groups are invariably from the mercantile
castes. For example, the 10 richest people in India, according to Forbes
magazine are: Lakshmi Mittal (Baniya), Mukesh Ambani (Baniya), Azim
Premji (Lohana), Ruia brothers (Baniya), Savitri Jindal (Baniya), Gautam
Adani (Baniya), K.M. Birla (Baniya), Anil Ambani (Baniya), Sunil Mittal
(Baniya), Adi Godrej (Parsi). (Aakar Patel,
http://www.thehindu.com/todays-paper/tp-opinion/why-caste-persists-in-politics/article2940080.ece).

About 75\% of all people appointed to the post of president of the
Federation of Indian Chambers of Commerce and Industry, the largest and
oldest apex business organisation in India, since it was founded in
1927, belong to mercantile castes, while over 85\% are upper-caste
Hindus. (http://ficci.in/about-Past-President.asp)

The persistence of casteist attitudes and relationships bolsters the
authority and political clout of these groups. On the other hand,
Dalits, Adivasis and most OBC castes lack the inherited wealth and
property, as well as the influential caste networks of the upper-castes.
(Harriss-White)

\section{Purpose of this Study}\label{purpose-of-this-study}

This study aims to shed light on the role played by caste in the
functioning of India's retail economy. It looks at some historical and
sociological dimensions of the evolution and structure of India's
markets and demonstrates how the caste system remains a pervasive force
in the Indian retail sector. It also considers the role affirmative
action can play in the creation of a more equitable and diverse retail
market, by looking at the effects of such policies in other countries

\section{Literature Review}\label{literature-review}

Many historians and economists have documented the past and present of
India's trade economy, however hardly of them any have paid sufficient
attention to how the caste system has played a defining role in the
evolution of Indian markets. On the few occasions when a specific
trading caste has been studied, there has been a tendency to glorify the
``entrepreneurial spirit'' of the caste (Timberg) without properly
interrogating how the exclusionary forces of the caste system have
helped propel certain caste groups to success in business at the expense
of others. Nonetheless, many of these studies, if engaged with
critically, provide insight into the evolution and functioning of the
trade economy.

Harish Damodaran's book `India's New Capitalists' traces the history of
India's trading castes, along with the trajectories of the newly
emerging entrepreneurial communities, from the colonial period to the
present day. Due to its focus on the histories of specific caste groups,
it often brings to light the vital role caste-capital plays in ensuring
success in business, and is an important source of information for this
project.

Barbara Harriss-White et al in their book `Dalits and Adivasis in
India's Business Economy' present an extensive analysis of the
disproportionately low participation rates of these two groups in
different spheres of the Indian economy over the last three decades,
using data from the economic censuses. They also investigate the factors
which prevent dalits and adivasis from undertaking private enterprise.

The well-known economist, journalist and intellectual Chandra Bhan
Prasad has done extensive research on the effects of neoliberalism on
India's lower castes, and his two research papers: `Markets and Manu:
Economic Reforms and its Impact on Caste in India' and `Rethinking
Inequality: Dalits in Uttar Pradesh in the Market Reform Era' present
many new insights into how Indian social relationships are being altered
in the face of rapid economic change.

The collection of research articles on caste discrimination, `Blocked by
Caste: Economic Discrimination in Modern India' edited by Sukhdeo Thorat
and Katherine S. Newman provides an authoritative and quantitatively
rigorous account of exclusion in various sectors and aspects of India's
economy. Through painstaking empirical work it demonstrates the
startling levels of discrimination faced by lower castes and minorities
in both market and non-market institutions.

The historical evolution of the Indian trade economy can be better
understood with the help of C.A. Bayly's acclaimed book `Rulers,
Townsmen and Bazaars', which gives a detailed account of economic
changes with occurred in the period between 1770 and 1870 in North
India. This period marks the downfall of the Mughal empire and the
beginning of British rule and was a period of great political and
economic upheaval. The book looks at how this period of transition
affected trade and trading communities in the cities, towns and qasbahs
of the gangetic plains, and is particularly insightful because it often
provides a caste-based perspective.

Claude Markovits in his book `Merchants, Traders and Entrepreneurs:
Indian Business in the Colonial Era', which is a collection of essays,
provides important insights into the evolution of Indian business and
entrepreneurship in the British era. He presents a history of the rise
of pan-Indian merchant networks, the rise of nationalism among the
Indian business class and the close symbiotic relationship it shared
with the Indian National Congress.

\section{Research Design}\label{research-design}

In the first part of this study I will present a historical account of
the evolution of Indian trade through the ages, with a focus on how this
evolution was driven by the exclusionary and exploitative ideology of
caste. In the second part I will analyse the present structure and
functioning of the Indian retail market based on economic data from
secondary sources such as the Economic Census and the National Sample
Survey, as well as earlier studies in this area. Finally, I will present
and analyse primary data collected through sample surveys from two small
town bazaars, both located in very different social, economic and
demographic contexts, to reach some conclusions about the caste
composition of retail markets in India.

The fieldwork for this study has been carried out in Karanjia block of
Dindori district in Madhya Pradesh, and in the city of Wardha,
Maharashtra. Karanjia is a rural, predominantly adivasi area located on
the Eastern border of Madhya Pradesh. It has a population of 86,802 of
which 74.3\% belong to the ST category and 3.2\% to the SC category. The
most populous adivasi groups in this region are Gond and Baiga. Wardha
is a small town in eastern Maharashtra which is also the headquarters of
Wardha district. It has a population of 1,06,444, of which 15.4\% belong
to the SC category and 5.5\% to the ST category. (2011 Census)

The fieldwork involved conducting sample surveys of shopkeepers in
retail markets at these two locations, with the primary objective of
determining the caste composition of the bigger and wealthier retailers
in these markets. The shops chosen were permanent medium and large-sized
retail establishments selling products such as groceries, jewellery,
clothing, medicine, electronics, utensils, furniture and cosmetics.
Given the nature of the survey and the paucity of time, carrying out
systematic random sampling was difficult, and so the subjects were
chosen arbitrarily during a stroll through the market, with an effort to
pick participants as randomly as possible. The sample size for the
Karanjia markets was 69 shopkeepers, while in Wardha 50 shopkeepers were
surveyed.

\section{Hypothesis}\label{hypothesis}

My hypotheses regarding the caste composition of a typical Indian bazaar
are:

\begin{itemize}
\item
  Most of the bigger retail establishments are owned by members of caste
  groups whose traditional occupation is trade.
\item
  Even though the caste composition of the trade economy is undergoing
  change, lower caste groups, especially dalits and adivasis have very
  low participation rates in private retail enterprise.
\item
  The Indian retail economy continues to be regulated to a large extent
  by the rules of caste, with every caste having a designated place and
  role.
\end{itemize}
\end{document}
