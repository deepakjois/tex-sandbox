\documentclass[a4paper]{article}

\usepackage[margin=1in,centering]{geometry}
\usepackage{fontspec}
\usepackage{array}
\usepackage{psvectorian}
\usepackage{bidi}

\newfontfamily\notourdu[Script=Arabic,Language=Urdu,WordSpace=5]{Noto Nastaliq Urdu}
\newfontfamily\itfhindi[Script=Devanagari]{ITF Devanagari}
\newfontfamily\baskerville{Baskerville}
\newfontfamily\titlefont[Script=Arabic,Language=Urdu]{Amiri Bold}

\newcommand{\deco}{\psvectorian[height=0.75cm]{71}}
\newcommand{\subtitle}[1]{\begingroup\fontsize{24pt}{12pt}\titlefont #1\endgroup\nopagebreak}

\begin{document}
% ---------------- COVER PAGE BEGINS  ----------------
\thispagestyle{empty}
\begin{center}
\itfhindi
\Huge \textbf{महात्मा गांधी अंतरराष्ट्रीय हिंदी विश्वविद्यालय, वर्धा}
\vskip 20pt
\huge भारतीय एवं विदेशी भाषा प्रगत अध्ययन केन्द्र
\vskip 10pt
भाषा विद्यापीठ
\vskip 42pt
\deco\deco\deco
\vskip 42pt
\huge \textbf{टर्म पेपर}
\vskip 20pt
\vskip 10pt
कहानी क़िस्से की
\vskip 10pt
\RL{\notourdu کہانی قصّے کی}
\vskip 42pt
\deco\deco\deco
\end{center}
\vskip 42pt
\itfhindi
\setlength{\tabcolsep}{0pt}
\noindent\begin{tabular}{>{\noindent\centering}p{225pt}>{\noindent\centering}p{225pt}}
\huge \textbf{प्रस्तुतकर्ता}
\vskip 10pt
\LARGE सुरभि अग्रवाल\\
उर्दू भाषा में एडवांस्ट डिप्लोमा\\प्रथम छमाही\\2015-16
&
\huge \textbf{मार्गदर्शक}\\
\vskip 10pt
\LARGE सेराज अहमद अंसारी\\
सहायक प्रोफेसर\\
भारतीय एवं विदेशी भाषा प्रगत अध्ययन केन्द्र\\
\end{tabular}
\clearpage
% ---------------- COVER PAGE ENDS ----------------
\newpage\null\thispagestyle{empty}\newpage
% ---------------- PAGE INTENTIONALLY LEFT BLANK ----------------
\pagenumbering{arabic}
\RTL
\setlength{\parindent}{0pt}
\setlength{\parskip}{24pt plus 10pt minus 10pt}
\begin{center}
{\fontsize{36pt}{18pt}\titlefont کہانی قصّےکی}

{\fontsize{10pt}{18pt}\notourdu
 سربھی اگروال\\
اردو زبان میں ایڈوانسڈ ڈپلوما ۲۰۱۵–۱۶\\
مہاتما گاندھی انتراشٹریہ ہندی وشوودیالۓ، وردھا}

\deco
\end{center}
\fontsize{12pt}{12pt}\linespread{2.5}\notourdu
قصّہ اردو نثر کی ایک صنف ہے۔ عربی کے لفظ ’قصّہ‘ اور فارسی کے  لفظ  ’داستان‘، دونو کا ہی معنی  کہانی ہوتا ہے۔ لیکن ان دونوں  الفاظ کو نثر کی ایک خاص صنف سے جوڑا جاتا ہے، جس کے تحت خیالی واقعیات بیان کرنے والی رومانی کہانیاں آتی ہیں۔  ایک داستان میں عام طور پر بہت سے طلسمی اور عجیب و غریب واقعات ہوتے ہیں جو سننے والے کو رہسی ، حیرت اور مسرت کا سامان فراہم کرتے ہیں۔

\subtitle{ قصّے کے عناصر اور داستان گوئی کا فن}

اردو کے سب سے مقبول قصوں پر نظر ڈالی جائے تو یہ پتا چلتا ہے کہ ان میں اکثر ایک ہیرو ہوتا ہے، جو ایک خوبصورت، بہادر، شریف اور اونچے خاندان کا آدمی ہوتا ہے۔ اکثر ہی وہ ایک اُتنی ہی خوبصورت اور شریف  عورت سے اشق کرتا ہے۔ وہ اپنی محبوبا کو پانے کے لئے بہت سی چنوتیوں کا سامنا کرتا ہے۔ وہ ڈراونے دیو، مخالف جادوگر اور خونخوار مخلوق کے ساتھ جنگ کرتا ہے۔ ان چنوتیوں میں اسے پریوں، جِنّوں، پِیروں، فقیروں اور دیگر چمتکاری طاقتوں  سے مدد ملتی ہے۔ کچھ ایسے واقعیات ہیں جن کا ہمیں کئی الگ الگ قصّوں میں استعمال ملتا ہے۔ قصّوں میں دوہراو  اور مشابہت عام ہوتے ھیں، لیکن ہر قصّے کی اپنی انفرادیت بھی ہوتی ہے۔

سترویں صدی میں فارس سے ہندستان آئے ابد النبی فخر الزمانی جو جہانگیر کے دور میں ایک جانے مانے داستان گو تھے، نے اپنی کتاب ’طراز الاخبار‘ میں داستان کے چار احم عناصر بتائے ہیں۔ یہ ہیں: ’رزم‘، ’بزم‘، ’حسن و عشق‘ اور ’ایّاری‘۔ یعنی جنگ ،  درباری شان و شوکت، خوب صورتی اور محبت، اور  دھوکہ بازی اور فریب، یہ سبھی عام طور پر ایک قصے میں پائے جاتے ہیں۔ ۱۸۵۵ میں ’داستانِ امیر حمزہ‘ کا اردو ترجما کرنے والے مصنف مرزا امن علی خان لکھنوی نے اس فہرست میں ایک اور عنصر جوڑا، جو تھا ’طلسم‘ یعنی جادوی کارنامے۔

الزمانی کی کتاب طراز الاخبار داستان گو کو اپنے فن پر ہدایات دینے کے مقصد سے لکھی گئی تھی۔ اس میں الزمانی نے قصے کو ایک’دروغ‘ یعنی جھوٹ کہا ہے اور ”حقیقت کے زیور سے محروم“ (”از حلیا صدق محروم“) کہا ہے۔  لیکن ساتھ میں انہونے سامعین پر داستان گوی  کے مثبت اثرات کی بھی بات کی ہے۔ ان کے مطابق داستان گو کی زبان کی یہ خصوصیت ہے کہ وہ ’فصیح‘ ، ’بالغ‘ اور ’روزمرا‘ ہوتی ہے، جس کو سننے سے سامعین کی لسانی سمجھ اور تلفظ میں اضافہ آتا ہے۔ دنیا اور ملک کے ساماجی اور سیاسی سلسلوں  سے جوجھیے کی نصیحت ملتی ہے، اور سننے والے میں ’تدبیر‘ یعنی حکمت اور صبر پیدا ہوتا ہے۔

الزمانی نے ایک داستان گو کے فن کی  وضاحت  کرتے ہوئے کہا ہے کہ وہ اپنے اشاروں اور اداؤں کے استعمال سے سامعین کو ان کی نظرِ تصوّر کے ذریعے داستان کے واقعیات سے مشاہد کرواتا ہے۔ مثلاً اگر قصّے کا کوئی قردار قید سے آزاد ہوتا ہے تو داستان گو ایسے پیش آتا ہے  جیسے وہ خود قید سے آزاد ہوا ہو۔ اس طرح وہ قصّے کے قردار  کو سامعین کے سامنے زندہ کر دیتا ہے۔ 

\subtitle{ہندوستان میں قصّے کی تاریخ}

قصے کی شروعات فارس کی عوام کے بیچ کہانیاں زبانی سننے سنانے کی روایت سے ہوئی۔ وقت کے ساتھ قصّوں کے بیان  میں ایک خاص قسم کی اداکاری کی جانے لگی اور داستان سنانے کے اس فن کو داستان گوئی یا قصّہ خوانی  کہا  گیا۔ قرون وسطی کے دور میں فارس میں پیشے ور داستان گو لوگوں کے بیچ مقبول ہوا کرتے تھے۔ آگے چل کر شاہی درباروں میں داستانیں سنائی جانے لگیں اور بادشاہوں کے قافلوں میں بھی داستان گو شامل ہونے لگے۔ جب ہندوستان میں فارسی بولنے والے بادشاہوں کی ریاستیں قایم ہوئیں تو داستان بھی فارس سے ہندوستان آئی۔

ایک قدیم اور ہندوستان میں بیحد مشہور داستان کا نام ہے ’قصّہِ حمزہ‘ یا ’داستانِ امر حمزہ‘۔ یہ داستان گیاروی صدی سے ہندوستان میں جانی جاتی ہے۔ گولکونڈا قطب شاہی دربار میں اس کو سنایا جاتا تھا۔ شہنشاہ اکبر کو بھی یہ بہت پسند تھی اور کہا جاتا ہے کہ وہ اس کو اپنے حرم میں خود سنایا کرتے تھے۔ انہونے اس داستان کے واقعات کی ١٤٠٠ تصویریں بھی بنوائیں۔

 اٹھاروی صدی میں اس قصّے کی ایک ہندوستانی نقل ’بوستانِ خیال‘  لکھی گئی۔ اس قصّے کی مقبولیت کا ایک اور ثبوت مشہور شاعر غالب کے اپنے ایک دوست کو فارسی میں لکھے خط میں ملتا ہے، جس میں وہ کہتے ہیں :

\begin{quote}
”پچاس ساٹھ جز کی کتاب داستان امیر حمزہ کی جلد آئی ہے اور اسی قدر حجام کی ایک جلد بوستان خیال کی آئی ہے۔ سترہ بوتل بادہ ناب توشک خانہ میں موجود ہے۔ دن بھر کتاب دیکھا کرتے ہیں ، رات بھر شراب پیا کرتے ہیں۔ زندگی سے اور کیا مراد رکھیں؟ ایسی خماری میں یہی خیال آتا ہے : جمشید کو اس سے زیادہ کیا نصیب ہوا ہوگا ؟ سکندر کو کیا نصیب ہوا ہوگا؟“
\end{quote}

فارسی قصّوں کی مقبولیت کی وجہ سے جلد ہی اردو میں بھی قصّے لکھ جانے لگے۔ سب سے پرانے اردو قصّے دکّنی اردو کے  ادب میں ملتے ہیں۔ ’سب رس‘ (۱۶۳۰) اور ’خوار نامہ‘ (۱۶۴۹) اس کی کچھ مثالیں ہیں۔

اتری ہندوستان میں اردو ادب اٹھاروی صدی میں ہی جا کر بالغ ہوا، یہاں پہلی بار اردو قصّے فورٹ ولیم کالج کلکتہ میں لکھے گئے۔ ان میں شامل تھے ’قصّہ چہار درویش‘، ’قصّہ گل و ہرمز‘، ’قصّہ گل و صنوبر‘، ’قصّہ گل بکاولی‘، ’قصّہ حاتم طائ‘ اور خلیل علی خان اشک کی ’داستانِ امیر حمزہ‘ جو دکّنی کی ’قصّہِ جنگِ امیر حمزہ‘ پر مبنی تھی۔

اردو قصّوں کی تخلیق ہونے سے اتری ہندوستان کی عوام میں داستان گوئی کی روایت کا آغاز ہوا۔ دہلی  میں  پیشےور داستان گوئی کی عام لوگوں کے بیچ نمایندگی ١٨٣٠ میں شروع ہوئی ۔ جامع مسجد کی سیڑھیوں پر ہر  جمعرات کو داستان گوئی سننے ہزاروں کی تعداد میں لوگ اکٹھے ہوتے۔ اس دور کے  دہلی، لکھنؤ اور رامپور کے کئی  مشہور داستان گو کا ذکر تاریخی دستاویزات میں  ملتے ہیں۔

قریبن ایک صدی تک اتری ہندوستان میں داستان گوئی عوام کے بیچ اور ادبی اور اشرفی لوگوں میں بھی مقبول رہی۔ ١٨٥٨ میں منشی نول کشور نے لکھنؤ میں نول کشور پریس شروع کی اور بہت سے اردو قصّوں کو چھاپنا شروع کیا۔ ٤٦ جلد کی ’داستان امر حمزہ‘ کو انہونے ١٨٩٣ اور ١٩٠٨ کے بیچ میں چھپا۔

انیسوی صدی کے آخر میں اردو ادب میں ناول کی اہمیت بڑھنے لگی اور اس کے ساتھ نثر میں قدرتی اور حقیقی اظہار کو رومانی اور طلسمی کہانیوں سے زیادہ توجہ دی جانے لگی۔ اس وجہ سے قصّوں کی روایت میں دھیرے  دھیرے زوال آنے لگا۔ لیکن اردو اور ہندی دونوں کی ابتدائی ناول پر قصّے کی صنف کے اثرات بھی دکھائی پڑتے ہیں۔

 آج ہندوستان میں داستان گوئی کا فن تقریبن ختم ہو چکا ہے لیکن عام لوگوں میں آج بھی قصّوں کی یاد باقی ہے۔ علاءالدين‎ کے چراغ، سندباد کی کشتی، وکرم بیتال اور طوطا میںآ کی کہانی، آج بھی عوام کے تصؤر کا حصہ  ہیں۔ جدید اردو، ہندی، پنجابی اور دیگر ہندوستانی ادب پر اس قدیم صنف کے اثرات آج بھی نظر آتے ہیں۔

\begin{center}
\deco

\subtitle{حوشی}
\end{center}

[۱] فرانسس پرچیٹ۔
\LR{\baskerville Marvelous Encounters: Folk Romance in Urdu and Hindi}، ۱۹۸۵۔\\
\LR{\baskerville \em{http://www.columbia.edu/itc/mealac/pritchett/00litlinks/marv\_qissa}}

[۲] پاشا م. خان۔
\LR{\baskerville A Handbook for Storytellers: The Tiraz Al-Akhbar and the Qissa Genre}\\
\LR{\baskerville Tellings and Texts: Music, Literature and Performance in North India, Open Book Publishers} میں شائع کیا گیا،  ۲۰۱۵۔

[۳] محمد رزا علی۔   اردو اصناف نثر کا مختصر تعارف،  ۱۵ ستمبر ۲۰۱۵۔\\
\LR{\baskerville \em{http://www.zabaan.com/blog/a-brief-introduction-to-urdu-prose-genres/}}

[۴] بی. اِن. گوسوامی۔
\LR{\baskerville The Lost Art of Story Telling}\\
\LR{\baskerville The Tribune}
میں شائع کیا گیا،  ۶ مارچ ۲۰۱۱۔\\
\LR{\baskerville \em{http://www.tribuneindia.com/2011/20110306/spectrum/art.htm}}

\end{document}
