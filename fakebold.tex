\documentclass[a4paper]{article}

\usepackage[margin=1in,centering]{geometry}
\usepackage{fontspec}
\usepackage{bidi}

\newfontfamily\notourdu[Script=Arabic,Language=Urdu,WordSpace=5]{Noto Nastaliq Urdu}
\newfontfamily\notourdubold[Script=Arabic,Language=Urdu,WordSpace=5,FakeBold=2]{Noto Nastaliq Urdu}

\begin{document}
\RTL
\setlength{\parindent}{0pt}
\setlength{\parskip}{24pt plus 10pt minus 10pt}
\fontsize{36pt}{18pt}
\begin{center}
\fontsize{36pt}{18pt}\notourdubold
دیپک جوئس

\fontsize{12pt}{18pt}\notourdu
قصّہ اردو نثر کی ایک صنف ہے۔ عربی کے لفظ ’قصّہ‘ اور فارسی کے  لفظ  ’داستان‘، دونو کا ہی معنی  کہانی ہوتا ہے۔ لیکن ان دونوں  الفاظ کو نثر کی ایک خاص صنف سے جوڑا جاتا ہے، جس کے تحت خیالی واقعیات بیان کرنے والی رومانی کہانیاں آتی ہیں۔  ایک داستان میں عام طور پر بہت سے طلسمی اور عجیب و غریب واقعات ہوتے ہیں جو سننے والے کو رہسی ، حیرت اور مسرت کا سامان فراہم کرتے ہیں۔
  دیپک جوئس
\end{center}
\end{document}
