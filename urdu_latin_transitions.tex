\documentclass[a4paper]{article}


\usepackage[centering,margin=1in]{geometry}
\usepackage{ucharclasses}
\usepackage{fontspec}
\usepackage{bidi}


\newfontfamily\notourdu[Script=Arabic,Language=Urdu,WordSpace=5]{Noto Nastaliq Urdu}
\newfontfamily\ufb[Script=Arabic,Language=Urdu]{Amiri}
\newfontfamily\amiri[Script=Arabic,Language=Urdu]{Amiri Bold}

% Noto Nastaliq does not have glyphs for Latin alphabet and numerals
% \setTransitionsForLatin{\begingroup\urdufallback}{\endgroup}

\begin{document}
\RTL
\begin{center}
\fontsize{36pt}{12pt}\amiri
سُربھی اگروال
\end{center}
\setlength{\parindent}{0pt}
\setlength{\parskip}{24pt plus 10pt}
\fontsize{12pt}{12pt}\linespread{2.5}\notourdu
قصّہ نثر کی ایک صنف ہے۔ عربی کے لفظ ’قصّہ‘ اور فارسی کے  لفظ  ’داستان‘، دونو کا ہی معنی  کہانی ہوتا ہے۔ لیکن دونوں ہی الفاظ کو نثر کی ایک خاص صنف سے جوڑا جاتا ہے، جس کے تحت خیالی واقعیات بیان کرنے والی رومانی کہانیاں آتی ہیں۔  

سترویں صدی میں فارس سے ہندستان آئے ابد النبی فخر الزمانی جو جہانگیر کے دور میں ایک جانے مانے داستان گو تھے، نے اپنی کتاب ’طراز الاخبار‘ میں داستان کے چار احم عناصر بتائے ہیں۔ یہ ہیں: ’رزم‘ ’بزم‘ ’حسن و عشق‘ اور ’ایاری‘۔ یعنی جنگ ، درباری شان و شوکت، خوب صورتی اور محبت، اور  دھوکہ بازی اور فریب، یہ سبھی عام طور پر ایک قصے میں پائے جاتے ہیں۔ 1900 میں داستانِ امیر حمزہ کا اردو ترجما کرنے والے مصنف مرزا امن علی خان لکھنوی نےاس فہرست میں ایک اور جز جوڑی، جو تھی ’طلسم‘  یعنی جادوی کارنامے۔

میرا نام انگریزی میں
\LRE{\ufb Deepak Jois is the name}
 لیخھا جاتا ہے۔ میں 
1923 میں پیدا ہوا تھا۔
میرا (میر،کرو)    آؤ گے
( *، \&، \% )


هذا نص بالعربية وفيه بعض الرموز الاخرى مثلا… (-، * ، + ، 23 ‪>‬ ،)

الزمانی کی کتاب طراز الاخبر داستان گو کو اپنے فن پر ہدایات دینے کے مقصد سے لکھی گئی تھی۔ اس میں الزمانی نے قصے کو ایک ’دروغ‘ یعنی جھوٹ کہا ہے اور ”حقیقت کے زیور سے محروم“ (”از حلیا صدق محروم“) کہا ہے۔  لیکن ساتھ میں انہونے سامعین پر داستان گوی  کے مثبت اثرات کی بھی بات کی ہے۔  داستان گو کی زبان کی یہ خصوصیت ہے کہ وہ ’فصیح‘ ، ’بالغ‘ اور ’روزمرا‘ ہوتی ہے، جس کو سننے سے سامعین کی لسانی سمجھ اور تلفظ میں اضافہ آتا ہے۔ دنیا اور ملک کے ساماجی اور سیاسی سلسلوں  سے جوجھیے کی نصیحت ملتی ہے، اور سننے والے میں ’تدبیر‘ یعنی حکمت اور صبر پیدا ہوتا ہے۔

ایک قصے میں اکثر ایک ہیرو ہوتا ہے جو ایک خوبصورت، بہادر، شریف اور اونچے خاندان کا آدمی ہوتا ہے۔ اکثر ہی وہ ایک اتنی ہی خوبصورت اور شریف  عورت سے اشق کرتا ہے۔ وہ اپنی محبوبا کو پانے کے لئے بہت سی چنوتیوں کا سامنا کرتا ہے۔ وہ ڈراونے دیو، مخالف جادوگر اور خونخوار ارو طلسمی مخلوق کے ساتھ جنگ کرتا ہے۔ ان چنوتیوں میں اسے پریوں، جِنوں ، پِیروں ، فقیروں اور دیگر چمتکاری طاقتوں  سے مدد ملتی ہے۔

لہزا، ایک داستان میں کئی طلسمی اور عجیب و غریب واقعات ہوتے ہیں جو سننے والے کو رہسی ، حیرت اور مسرت کا سامان فراہم کرتے ہیں۔

قصے کی شروات فارس کی عوام کے بیچ کہانیاں زبانی سننے سنانے کی روایت سے ہوئی۔ قرون وسطی کے دور میں فارس میں پیشے ور داستان گو لوگوں کے بیچ مقبول ہوا کرتے تھے . آگے چل کر شاہی درباروں میں داستانیں سنائی جانے لگیں اور بادشاہوں کے قافلوں میں بھی داستان گو شامل ہونے لگے . جب ہندوستان میں فارسی بولنے والے بادشاہوں کی ریاستیں قایم ہوئیں تو داستان بھی فارس سے ہندوستان آئی.

ایک قدیم اور ہندوستان میں بیحد مشهور داستان کا نام ہے قصّہ حمزہ یا داستان امر حمزہ . یہ داستان گیاروی صدی سے ہندوستان میں جانی جاتی ہے. گولکنڈا قطب شاہی دربار میں اسکو سنایا جاتا تھا . شہنشاہ اکبر کو بھی یہ بہت پسند تھی اور کہا جاتا ہے کہ وہ اسکو خود اپنے حرام میں سنایا کرتے تھے . انہونے اس داستان کے واقعات کی ١٤٠٠ تصویریں بھی بنوائیں.

 ١٥ جلد کی داستان بھی لکھی گئی. اس قصّے کی مقبولیت کا ایک اور ثبوت مشہور شاعر غالب کے اپنے ایک دوست کو فارسی میں لکھے خط میں ملتا ہے، جسمیں وہ کہتا ہیں :

پچاس ساٹھ جز کی کتاب داستان امیر حمزہ کی جلد آئی ہے اور اسی قدر حجام کی ایک جلد بوستان خیال کی آئی ہے . سترہ بوتل بادہ ناب توشک خانۂ میں موجود ہے . دن بھر کتاب دیکھا کرتے ہیں ، رات بھر شراب پیا کرتے ہیں . زندگی سے اور کیا مراد رکھیں؟ ایسی خماری میں یہی خیال آتا ہے : جمشید کو اس سے زیادہ کیا نصیب ہوا ہوگا ؟ سکندر کو کیا نصیب ہوا ہوگا؟

وقت کے ساتھ قصّوں کے بیان  میں ایک خاص قسم کی اداکاری کی جانے لگی اور داستان سنانے کے اس فن کو داستان گوئی یا قصّہ خوانی  کہا  گیا .

فارسی قصّوں کی مقبولیت کی وجہ سے جلد ہی اردو میں بھی قصّے لکھ جانے لگے . سب سے پرانے اردو قصّے دکّنی اردو کے  ادب میں ملتے ہیں. سب رس جو ١٦٣٥ میں لکھی گی اور خوار نامہ ( جو داستان  امیر حمزہ سے متاثر تھی ) اس کی کوکچھ مثالیں ہیں .

 یہاں پہلی بار اردو قصّے فورٹ ولیم کالج کلکتہ میں لکے گئے . ان میں شامل تھے قصّہ چہار درویش، قصّہ گل و ہرمز ، قصّہ گل و صنوبر، قصّہ گل بکاولی، قصّہ حاتم طائ اور خلیل علی خان اشک کی داستان امیر حمزہ جو دکّنی قصّہ جنگ امیر حمزہ پر مبنی تھی. 

اردو قصّوں کی تخلیق ہونے سے اتری ہندوستان کی عوام میں داستان گوئی کی روایت کا آغاز ہوا . دہلی  میں  پیشے ور داستان گوئی کی عام نمایندگی ١٨٣٠ میں شرو ہی . جامہ مسجد کی سیڑھیوں پر ہر  جمعرات کو داستان گوئی سننے ہزاروں کی تعداد میں لوگ اکٹھے ہوتے . اس دور کے  دہلی، لکھنؤ اور رامپور کے کئی  مشهور داستان گو ک ذکر تاریخی دستاویزات میں  ملتے ہیں .

قریبن ایک صدی تک اتری ہندوستان میں داستان گوئی عوام کے بیچ اور ادبی اور اشرفی لوگوں میں بھی مقبول رہی . ١٨٥٨ میں منشی نول کشور نے لکھنؤ میں ناول کشور پریس شرو کی اور بہت سے اردو قصّوں کو چھاپنا شرو کیا. ٤٦ جلدوں کی داستان امر حمزہ کو انہونے ١٨٩٣ اور ١٩٠٨ کے بیچ میں چھپا .

الزمانی نے ایک داستان گو کے فن کی  وضاحت  کرتے ہوئے کہا ہے کہ وہ اپنے اشاروں اور اداؤں کے استعمال سے سامعین کو انکی نظر تصوّر کے ذریعے داستان کے واقعیات سے مشاہد کرواتا ہے . مَثَلاً اگر قصّے کا کوئی قردار قید سے آزاد ہوتا ہے تو داستان گو ایسے پیش آتا ہے  جیسے وہ خود قید سے آزاد ہوا ہو . اس طرح وہ قصّے کے قردار  کو سامعین کے سامنے زندہ کر دیتا ہے . 

انیسوی صدی کے آخر میں اردو ادب میں ناول کی اہمیت بڑھنے لگی اور اسکے ساتھ نثر میں قدرتی اور حقیقی اظہار کو رومانی اور طلسمی کہانیوں سے زیادہ عزت دی جانے لگی . اسے قصّوں کی روایت میں دھیرے دھیرے زوال آنے لگا  آج ہندوستان میں داستان گوئی کا فن تقریبن ختم ہو چکا ہے لیکن عام لوگوں میں آج بھی قصّوں کی یاد باقی ہے .علاء الدين‎ کے چراغ ، سندباد کی کشتی ، وکرم بیتال اور طوطا میںآ کی کہانی، آج بھی مقبول تخیل کا حصہ  ہیں. جدید اردو، ہندی ، پنجابی اور دیگر ہندوستانی ادب پر اس قدیم صنف کے اثرات آج بھی نظر آتے ہیں .
\end{document}